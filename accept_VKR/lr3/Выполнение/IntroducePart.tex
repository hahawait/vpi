\section{Введение}

\paragraph*{} Текст можно рассматривать как сочетание темы и стиля. Тема определяет содержание текста, а стиль отражает особый способ 
автора манипулировать словами. Основная 
идея этой статьи состоит в том, чтобы изучить отдельное представление темы и пред- ставление стиля для данного текста. 
В этом исследовании предлагается многозадачный подход для совместной оптимизации основной задачи – атрибуции 
авторства и вспомогательной задачи – аппроксимации темы.
\par В частности, задача аппроксимации 
темы состоит в том, чтобы создать представление темы для аппроксимации распределения темы текста. 
Распределение тем определяется независимыми от задачи моделями, которые обучаются на внешнем корпусе текстов [1]. 
Таким образом, наша структура обеспечивает контроль для разделения стилей тем и не требует человеческого вмешательства 
для аннотирования данных.
\par Цель исследования – изучить способы определения авторства и аппроксимации темы текста.