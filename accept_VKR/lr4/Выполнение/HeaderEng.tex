
\begin{center}
\textbf{
\title{APPLICATION OF MULTITASK LEARNING TO DETERMINE TEXT 
AUTHORSHIP BASED ON MECHANISM OF COMPETITIVE ATTENTION }\\
\author{Baturin M.M., Belov Yu.S.}}
\end{center}
\maketitle
\begin{abstract}
    In the tasks of determining the authorship of a text, the presentation of the author’s personal style, independent 
    of the subject matter, plays a key role. Thus, separating the content of the text from the stylistic features of the 
    author’s writing is an important problem. To solve this problem, powerful unrealistic solutions are often used, 
    or manually defined text style parameters. This article proposes to apply multi-task learning to separate the topic 
    of the text from the style of the author. The goal of the proposed approach is to find separate representations of 
    the style and theme of the text. The main task is to determine the authorship of the text, an additional task is to 
    approximate the topic. The models used to obtain representations of topics are trained on an external data corpus. 
    The article proposes the mechanisms of competitive attention and split-recovery constraints, by which two tasks 
    are assigned different and competing attentions, which contributes to the separation of theme and style. Based on 
    the results of the assessments, the multitasking learning approach is promising, especially for a dataset with many 
    overlapping themes. The proposed model separates theme and style in a probabilistic way and does not require 
    human intervention.
\end{abstract}
\textbf{Keywords: text authorship attribution, approximation of the topic of the text, recurrent neural networks, competitive 
attention}