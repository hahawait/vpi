\title{\textbf{ПРИМЕНЕНИЕ МНОГОЗАДАЧНОГО ОБУЧЕНИЯ\\
    ДЛЯ ОПРЕДЕЛЕНИЯ АВТОРСТВА ТЕКСТА\\
    НА ОСНОВЕ МЕХАНИЗМА КОНКУРЕНТНОГО ВНИМАНИЯ}}
\author{\textbf{Батурин М.М., Белов Ю.С.}}
\date{} 
\maketitle

\textit{ФГБОУ ВО «Московский государственный технический университет имени Н.Э. Баумана», 
филиал, Калуга, e-mail: k4dys@yandex.ru}

% Аннотация
\begin{abstract}
    В задачах определения авторства текста ключевую роль играет представление независящего от тематики произведения личного стиля автора. Таким образом, отделение содержания текста от стилистических 
особенностей письма автора является важной проблемой. Для решения этой проблемы зачастую используются мощные нереалистичные решения, либо вручную определённые параметры стиля текста. В этой 
статье предлагается применить многозадачное обучение, чтобы отделить тему текста от стиля автора. Цель 
предложенного подхода состоит в том, чтобы найти отдельные представления стиля и темы текста. 
Основной задачей является определение авторства текста, дополнительной задачей является аппроксимация темы. 
Применяемые для получения представлений тем модели обучаются на внешнем корпусе данных. В статье 
предложены механизмы конкурентного внимания и ограничения разделения-восстановления, при помощи 
которых двум задачам назначаются разные и конкурирующие между собой внимания, что способствует разделению темы и стиля. По результатам оценок подход, основанный на многозадачном обучении, является 
многообещающим, особенно при наличии набора данных с множеством пересекающихся тем. Предложенная модель 
разделяет тему и стиль вероятностным образом и не требует вмешательства человека.
\end{abstract}
\textbf{Ключевые слова: определение авторства текста, аппроксимация темы текста, рекуррентные нейронные сети, 
конкурентное внимание}
